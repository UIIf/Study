\documentclass[14pt, titlepage,fleqn,a4paper]{extarticle}
\usepackage[T1,T2A]{fontenc}
\usepackage[utf8]{inputenc}

\usepackage{amsmath}
\usepackage[russian]{babel}

% \usepackage{titlepage}
\usepackage{listings}
\usepackage{color}
\usepackage{graphicx}
\usepackage{float} 

\usepackage{caption}
% \captionsetup[figure]{font=small}

\newcommand{\InsertGraf}[2]{
	\begin{figure}[H]
		\center{\includegraphics[width = \textwidth]{#1}}
		\caption{#2}
	\end{figure}
}

\definecolor{dkgreen}{rgb}{0,0.6,0}
\definecolor{gray}{rgb}{0.5,0.5,0.5}
\definecolor{mauve}{rgb}{0.58,0,0.82}


\lstset{frame=tb,
	language=Python,
	aboveskip=3mm,
	belowskip=3mm,
	showstringspaces=false,
	columns=flexible,
	basicstyle={\small\ttfamily},
	numbers=none,
	numberstyle=\tiny\color{gray},
	keywordstyle=\color{blue},
	commentstyle=\color{dkgreen},
	stringstyle=\color{mauve},
	breaklines=true,
	breakatwhitespace=true,
	tabsize=3
}

\title{Разработка прототипа игры жанра Top-Down Shooter на базе Unity}
\author{Uiif Ui}
\date{March 2022}

\begin{document}

    \maketitle
    %--------------------------------------------------------------------
	\tableofcontents   
	\setcounter{page}{1}
	\newpage
	%--------------------------------------------------------------------
	\section*{Введение}
	\addcontentsline{toc}{section}{Введение}
	В последнее время все большие обороты набирает гейм индустрия, и дело не только в том, что людям требуется интересное проведение досуга. Так же компьютерные игры могут быть использованны в сфере образрования, науки, промышленности и в других сферах деятельности человека. Кроме того сама по себе разработка игр - достаточно интересный вид деятельности. 
	
	Университет дает достаточно знаний для того, чтобы начать погружение в эту сферу. Задачей стало применение их на практике. 
	
    % В данной курсовой работе речь пойдет о создании прототипа будущего игрового проекта ‹‹That’s enough››, который и будет являться объектом исследования. 
    % В первой главе данной курсовой работы приводится аналитическая часть, в которой рассматриваются все аспекты проекта для будущей реализации.
    % Во второй главе приведены этапы разработки прототипа компьютерной игры с описанием реализованных в ходе работы алгоритмов.
    % В заключении приведены выводы по курсовой работе.
    Цель работы: создание прототипа будущего игрового проекта в жанре Top-down shooter, на базе Unity.
    
    Задачи:
    \begin{itemize}
        \item Познакомиться с игровой индустрией
        \item  Пердставить конечный вид проекта
        \item Подобрать необходимую теоретическую базу
        \item Написать прототип проекта
    \end{itemize}

    \section*{Первая часть}
	\addcontentsline{toc}{section}{Первая часть}
	
	%Что то про пользу игр
	%
	
	% ДИЗДОК
	Геймплей
Высокая скорость перемещения
Ограниченный боезапас для огнестрельного оружия (нет перезарядки, для оружия 2 ячейки, оружие выбрасывается и подбирается с пола новое)
Выброшенное оружие при попадании во врага оглушает его
Метательное оружие (гранаты)
Полоска здоровья, здоровье врагов увеличивается по мере прохождения
(?) Чтобы завершить игру игроку нужно найти несколько артефактов

Игровой мир
Генерация мира(комнат) с удалением блоков, находящихся далеко от игрока 

Игровая механика
Периодически случайным образом будут появляться комнаты с новым типом оружия, которое после взятия будет так же появляться у врагов

Интерфейс
Количество здоровья, ячейки оружия, ячейка метательного оружия, количество патронов в выбранном оружии, если оно огнестрельное.

Враги
Несколько типов врагов, отличающиеся количеством здоровья, внимательностью и скоростью перемещения
    % ДИЗДОК
	
	\section*{Вторая часть}
	\addcontentsline{toc}{section}{Вторая часть}
	
	\section*{Заключение}
	\addcontentsline{toc}{section}{Заключение}
    
\end{document}
